% The following homework template was modified from the Fall 2017 CIS 160 homework template.

\documentclass{article}
\usepackage[letterpaper,top=0.5in,bottom=0.5in,left=1.2in,right=1.2in,includeheadfoot]{geometry}
\usepackage{amssymb,amsmath}
\usepackage{parskip}
\usepackage{fancyhdr}
\usepackage{tabu}
\usepackage{enumerate}
\usepackage{graphicx}
\usepackage{tikz}
\usepackage{amsthm}

\renewcommand{\baselinestretch}{1.3}
\pagestyle{fancy}
\fancyfoot{}
\renewcommand{\headrulewidth}{0pt}
\setlength{\headheight}{0pt}

\fancypagestyle{firstpage}{
  \fancyhead{}
  \fancyfoot{}
}

% Modify here to change homework number, name, and PennKey as necessary.
\newcommand{\HomeworkNo}{}
\newcommand{\MyName}{Kevin Monogue}

\fancyhead[L]{\MyName}

\fancyhead[R]{Linear Impact Model for Trade Order Book}

\newcommand{\PrintFirstHeader}{
  CME 241 \vspace{5pt} \hfill {\Large{\MyName}}
  \\
  {\LARGE{\textbf{Linear Impact Model for Trade Order Book}}} 

  \rule{\textwidth}{0.4pt}}
  
\begin{document}
\thispagestyle{firstpage}
\PrintFirstHeader{}

\textbf{Problem Definition}\\
We have a big order of shares that we'd like to sell in predetermined amount of time. Given executing this order is likely to have price impact effects, we'd like to optimize a strategy in-order to maximize the utility of proceeds received from the sale. Naturally, this will entail balancing a trade off between certainty (ex: hitting the immediate bids all the way down at known prices) vs. pure value (ex: selling slowly to only ever hit the top bid). \\

\textbf{Variable Definitions and Dynamics} \\
N = number of shares to be sold \\
T = number of discrete time steps \\
$N_t$ = number of shares sold at time step t \\
$P_t$ = bid price at time step t \\
$R_t$ = shares remaining at time step t = $N - \sum_{i = 1}^{t-1} N_i$ \\
$P_{t+1} = f_t(P_t, N_t, \epsilon_t)$ - price is a function of previous price, shares sold, and randomness \\
$N_t \dot Q_t = N_t \dot (P_t - g_t(P_t, N_t))$ - proceeds received is equal to shares X (price - impact) \\
$U$ = utility function \\

\textbf{Simple Linear Model} \\
Price dynamics are modeled as a simple linear function with randomness, $P_t = P_{t-1} - \alpha N_{t-1} + \epsilon_{t-1}$ where $\epsilon_t$ is i.i.d. with zero expectation and $\alpha > 0$. This captures a long-term price impact effect. The price in the next timestep is dependent on the amount sold in the previous timestep.
Temporary price impact is applied through the proceed function. The amount received in this period is dependent on the amount sold in this period. Mathematically $Q_t = P_t - \beta N_t$ where $\beta > 0$. The effective price received is reduced by number of shares sold. \\

The value function for this model is relatively easily defined. Our state is a function of the current timestep, the current price, and the amount of shares remaining to be sold. The value of being at this timestep is the expected future proceeds we will receive, which is simply the sum across future time periods amount sold x effective received. Mathematically this looks like:
\begin{center}
$V^\pi(t, P_t, R_t) = E~[\sum_{i = t}^{T} N_i \cdot Q_i | (t, P_t, R_t)]  $ \\
$V^\pi(t, P_t, R_t) = E~[\sum_{i = t}^{T} N_i \cdot (P_i - \beta N_i | (t, P_t, R_t)]  $ 
\end{center}

We would like to solve for the optimal value function. We can begin by first stating that at any time step $t$, the optimal value function entails the optimal decision this period + the optimal value function at the next state.
\begin{center}
$V^*(t, P_t, R_t) = max_{N_{t}} N_t \cdot (P_t - \beta N_t) + E~[V^*(t + 1, P_{t+1}, R_{t+1})]$ 
\end{center}

Like with most methods, we know what the terminal value will be. We should simply sell the remaining shares for whatever we can get. Thus, we can back out the previous steps value function.

\begin{align*}
V^*(T, P_T, R_T) & = R_T \cdot (P_T - \beta R_T) \\
V^*(T-1, P_{T-1}, R_{T-1}) & = max_{N_{T-1}} N_{T-1} \cdot (P_{T-1} - \beta N_{T-1}) + E~[R_T \cdot (P_T - \beta R_T) ] 
\end{align*}
\begin{align*}
 = max_{N_{T-1}} N_{T-1} \cdot (P_{T-1} - \beta N_{T-1}) + E~[(R_{T-1} - N_{T-1}) \cdot (P_T - \beta (R_{T-1} - N_{T-1})) ] \\
 = max_{N_{T-1}} N_{T-1} \cdot (P_{T-1} - \beta N_{T-1}) + (R_{T-1} - N_{T-1}) \cdot (P_{T-1} - \alpha N_{T-1}) - \beta (R_{T-1} - N_{T-1})) \\
\end{align*}
 $= N_{T-1} \cdot P_{T-1} - \beta N_{T-1}^2 + R_{T-1} \cdot P_{T-1} - \alpha N_{T-1} \cdot R_{T-1}  - \beta R_{T-1}^2 + \beta R_{T-1}{N_T-1} - N_{T-1} \cdot P_{T-1} + \alpha N_{T-1}^2 + \beta R_{T-1} \cdot N_{T-1} - \beta N_{T-1}^2 $ \\
 $ = R_{T-1} \cdot P_{T-1} - \beta R_{T-1}^2 + (\alpha - 2 \beta) (N_{T-1}^2 - N_{T-1} R_{T-1}) $\\
 
Notice that if $ \alpha > 2 \beta$, we will want the second portion to be either negative or 0. The most we can sell is $R_{T-1}$, so it has to be 0. That means the optimal solution is either to sell 0 or to sell all of the remaining. In this case, we can plug this in to the optimal value function above and get 
\begin{center}
$V^*(T-1, P_{T-1}, R_{T-1}) = R_{T-1} \cdot (P_{T-1} - \beta R_{T-1})$
\end{center}
Notice this is the same as the final stage value function (as if we sold all our remaining shares in this stage). Thus, we can repeat this process for every prior stage and obtain $N_t = 0$ or $R_t$ and $V^*(t, P_t, R_t) = R_{t} \cdot (P_{t} - \beta R_{t})$. The optimal decision is to sell all shares in one of the time periods. Remember, $\alpha$ is the long term price impact, so when it is exceedingly high it makes the most sense to sell all your shares once you start selling. \\
The other case is when $\alpha < 2 \beta$. We can solve for the max of our original statement by differentiating with respect to $N_{T-1}$.
\begin{align*}
0 & = (a - 2\beta) \cdot (2 N_{T-1} - R_{T-1}) \\
N_{T-1} & = R_{T-1} / 2
\end{align*}
Again we plug this back in to the value function for this timestep...
\begin{align*}
V^*(T-1, P_{T-1}, R_{T-1}) & = R_{T-1} \cdot P_{T-1} - \beta R_{T-1}^2 + (\alpha - 2 \beta) (R_{T-1}^2 / 4 - R_{T-1} R_{T-1} / 2) \\
& = R_{T-1} \cdot P_{T-1} -  R_{T-1}^2 \cdot \frac{\alpha + 2\beta}{4}
\end{align*}
Counting backwards
\begin{align*}
V^*(T-2, P_{T-2}, R_{T-2}) & = max_{N_{T-2}} N_{T-2} \cdot (P_{T-2} - \beta N_{T-2}) + R_{T-1} \cdot P_{T-1} -  R_{T-1}^2 \cdot \frac{\alpha + 2\beta}{4} \\
& = max_{N_{T-2}} N_{T-2} \cdot (P_{T-2} - \beta N_{T-2}) + (R_{T-2} - N_{T-2}) \cdot (P_{T-2}  - \alpha N_{T-2}) -  (R_{T-2} - N_{T-2})^2 \cdot \frac{\alpha + 2\beta}{4} \\
0 & =  (P_{T-2} - 2\beta N_{T-2}) - \alpha R_{T-2} - P_{T-2} + 2\alpha N_{T-2} + \frac{\alpha + 2\beta}{2} (R_{T-2} - N_{T-2}) \\
& = -3 \beta N_{T-2} + 3/2 \alpha N_{T-2} - 1/2 \alpha R_{T-2} + \beta R_{T-2} \\
& = (3N_{T-2}) (1/2 \alpha - \beta) - R_{T-2} (1/2 \alpha - \beta) \\
N_{T-2} & = \frac{R_{T-2}}{3}
\end{align*}
Notice the pattern is to evenly distribute the remaining sales across remaining periods. From this we obtain $N_t^* = \frac{R_t}{T-t + 1}$ and 
\begin{align*}
V^*(t, P_t, R_t) & = R_t P_t - R_{t}^2 \cdot \frac{2\beta + (T - t)\alpha}{2 (T-t+1) }
\end{align*}

Note here that the optimal policy is to uni formally split and does not depend on the price when at any given state. When the long term and short term price impact do not have any interaction effects, such as in this case, this makes intuitive sense. You want to minimize the damage done across the entire timespan, where the damage down in any single period is proportional to the amount sold in that timespan. Thus an even split minimizes the sum of these damages. As a note, the expected optimal proceeds is 
\begin{center}
$NP_1 - \frac{N^2}{2} (\alpha + \frac{2\beta - \alpha}{T})$
\end{center}
The shortfall is the second half of that equation (the potential profits missed out on due to the price impact. Notice this is inversely proportional to the time, and proportional to the amount sold and the size of the price effects.



\end{document}
